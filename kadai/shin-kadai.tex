\documentclass[submit,uplatex]{ipsj}
%\documentclass{ipsj}

\usepackage{graphicx}
\usepackage{latexsym}

\def\Underline{\setbox0\hbox\bgroup\let\\\endUnderline}
\def\endUnderline{\vphantom{y}\egroup\smash{\underline{\box0}}\\}
\def\|{\verb|}

\setcounter{巻数}{59}
\setcounter{号数}{1}
\setcounter{page}{1}


\受付{2016}{3}{4}
\再受付{2015}{7}{16}   %省略可能
\再再受付{2015}{7}{20} %省略可能
\再再受付{2015}{11}{20} %省略可能
\採録{2016}{8}{1}




\begin{document}


\title{NECO課題04}

\etitle{How to not complete homework}

\author{山本 朋義}{Tomoyoshi Yamamoto}{SFC}[karolus@sfc.wide.ad.jp]

\begin{abstract}
現実世界でも応用できる強化学習エージェントを実現するのは、より良いアルゴリズムか、それとも技術の応用方法という合わせ技か。
\end{abstract}


\begin{jkeyword}
課題
\end{jkeyword}

\begin{eabstract}
This is just one of those do it yourself homeworks, use Google Translate on the Japanese abstract to save my time ty.
\end{eabstract}

\begin{ekeyword}
Homework
\end{ekeyword}

\maketitle

%1
\section{はじめに}

ゲーム世界には、プレイヤー意外に「ノンプレイヤーキャラクター」(通称:NPC)が存在する。プレイヤーが操作する「プレイヤーキャラクター」(通称:PC)とは対義的な存在である。

特にRPG系のゲームにおいて、NPCは現実にいそうな振る舞いをするものがさまざまな形で存在する。その例として、家族的なキャラクター、店の店員・主人、村人、敵(モンスター、魔王)であったりする。これらNPCはそれぞれの行動パターンや会話ダイアログが事前にマニュアル式でプログラムされている。機械的なレスポンスや環境への適応を超越するNPCを強化学習と自然言語処理を用いて実現することが可能ではないかと考えている。

ただ、強化学習の研究の方向性は二つあり、より一つまたは広範囲な問題に適応した、より良いアルゴリズムの追求(高校時代はこちらを少し興味本位でやってたやつ)か、応用方法を追求する、他の研究でもよく見える既存の技術を活用した研究開発的アプローチのいずれかになる。強化学習研究のよく見られる主軸は、前者の方であり、自分自身は後者のような研究を試みているが、過去のアプローチの応用例を見ることでしっかり考える機会になるのではないかと感じた。



%2
\section{アルゴリズムか手段か}

シンプルに強化学習は、いわゆるコンピューターシミュレーションを用いた反復学習である。すなわち、対戦ゲームであったり、反復学習をうまく応用できる事象において強化学習はうまく応用される。現実世界におけるタスクは数えられるほどのシナリオでできているものは少ない。他の一点収束的な学習手法とは違い、それぞれの事象を経験した、似たシナリオ・状況と照らし合わせるような学習ができる強化学習は、そんな複雑な現実世界のタスクをタックルする糸口に違いないと考えた。

そんな強化学習だが、高校時代に軽い気持ちで学んでいた時に感じたのは、主な研究の方向性は「何かを成し遂げるためにはどの強化学習アルゴリズムが良いか」、「アルゴリズムの精度をあげよう!」という内容のものが主軸であり、逆に複数の技術との組み合わせ、複数の強化学習エージェントの組み合わせによる合わせ技を用いた成果を述べるものはない。自分の研究は後者の方ではあるが、先行研究を照らし合わせて、結局はアルゴリズム一つを突き詰めるべきものなのか否かをまとめた。

\subsubsection{アルゴリズム}

一つの論文に、「営業活動における意思決定の補助」を二つの学習モデルを比較する形で研究に持ち込んでいる論文があった。この論文の内容とは、シンプルにいえば「過去のツイートからツイート内容、言葉の並べ方のパターンを読み取り、ツイッターアカウントの真似をするサービス」的なパターン分析アルゴリズムを従来は使っていたが、強化学習を使えばもっと良い成果が出るのか?という研究。マルコフ系のアルゴリズムを使った研究です。結果は強化学習の勝利らしい。アンケート調査で結果示してる系な論文だが、、。これが一番アルゴリズム探します、試します系の研究で参考になるわかりやすい論文だった。

そのほかにもOpenAIのGymを用いた、制度の比較をしている研究であったり、自動運転技術の発展など。

\subsubsection{手段、応用、そして先行研究の山}

一つの論文に差し掛かり、先行研究の山がたくさん出てきた。先行研究といえど、自分の研究と全く合致しているというわけではないが、それは自然的な人間的、反転的な会話をするシステムを強化学習を用いて開発するという研究分野であった。締め切り間近なので、全てを取り上げることはできないが、一番関連性があったのが「化学習を用いてキャラクタらしさを付与した雑談応答の生成」であった。この研究によると、自然言語あたりの実装も強化学習でより良い制度で実現することができる。営業活動の補助システムをマルコフから強化学習にするようなアプローチになる可能性あり。

\begin{thebibliography}{2}
\bibitem{book1}
清水 健吾, 上垣 貴嗣, 菊池 英明:
強化学習を用いてキャラクタらしさを付与した雑談応答の生成,
一般社団法人 人工知能学会(2022).

\bibitem{book1}
中山 義人, 森 雅広, 斎藤 忍, 成末 義哲, 森川 博之:
深層強化学習による営業活動意思決定支援システム,
情報処理学会論文誌(2022).

\end{thebibliography}



\end{document}
